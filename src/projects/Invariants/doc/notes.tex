% notes.tex
% notes on bam/src/projects/Invariants
% BB, JG, SH, US 3/2006

\documentclass[12pt]{article}

\voffset=-2.5cm
\hoffset=-1.0cm
\textheight=24.5cm
\textwidth=15cm

\newcommand{\beq}{\begin{equation}}
\newcommand{\eeq}{\end{equation}}
\newcommand{\bea}{\begin{eqnarray}}
\newcommand{\eea}{\end{eqnarray}}
\newcommand{\mysec}[1]{\bigskip\noindent{\bf #1}\smallskip}


\begin{document}

\begin{center}
{\large\bf Notes on Curvature Invariants in BAM}
\end{center}
\bigskip
\bigskip
\bigskip



\mysec{Definition of $\psi_4$}

The Weyl scalars $\psi_i$, $i = 0$ through $4$, are defined by
contraction of the Weyl tensor $C_{abcd}$ with a specific combination
of vectors forming a null tetrad $\{l^a,n^a,m^a,\bar{m}^a\}$. Here we
consider
%
\begin{equation}
        \psi_4 = C_{abcd} n^a \bar{m}^b n^c \bar{m}^d .
\end{equation}
%
Note: This definitions presents us with the first of a number of
sign ambiguities. Fiske et.\,al\,\cite{Fiske2005},
for example, define $\psi_4$ as the negative of this expression.
As a matter of fact, the negative version is currently implemented
in the {\sc Bam} code.

As the Riemann tensor itself is subject to sign conventions, we
explicitly give our definition
%
\begin{equation}
  R^\alpha{}_{\beta \gamma \delta} = \partial_\gamma
        \Gamma^\alpha_{\beta \delta} - \partial_{\delta}
        \Gamma^\alpha_{\beta \gamma} + \Gamma^\alpha_{\gamma \mu}
        \Gamma^{\mu}_{\beta \delta} - \Gamma^\alpha_{\delta \mu}
        \Gamma^\mu_{\beta \gamma}.
\end{equation}
%
Note: As seems to be the practice in other vacuum codes, too, we
use the Riemann tensor and not the Weyl tensor as they are identical
if the Einstein equations (Ricci equal to zero) are satisfied.  

Note: This expression can be computed for any tetrad, but $\psi_4$ is
defined using a special tetrad to obtain standard
properties. (``principal null directions''?)

Note: Explain ``outgoing radiation property''.



\mysec{The tetrad}

Various definitions are in use for $\{l^a,n^a,m^a,\bar{m}^a\}$. The
vectors $l^a$ and $n^a$ are real while for convenience $m^a$ is chosen
to be complex with its complex conjugate denoted by $\bar{m}^a$. The
defining properties of a standard null tetrad are that (i) the vectors
are null, (ii) a specific normalization is assumed. Following
Chandrasekhar \cite{Chandrasekhar1983}:
%
\begin{eqnarray}
        && l^a l_a = n^a n_a = m^a m_a = \bar{m}^a\bar{m}_a = 0 \\
        && l^a n_a =-1,\quad m^a \bar{m}_a = 1,
        \quad \mbox{all other pairings 0} .
\end{eqnarray}
%

Note: This definition depends on the metric. In the far field of an
asymptotically flat space time we can use the following standard null
tetrad of Minkowski space:
%
\begin{eqnarray}
        l^a &=& \frac{1}{\sqrt{2}} (t^a + r^a), \\
        n^a &=& \frac{1}{\sqrt{2}} (t^a - r^a), \\
        m^a &=& \frac{1}{\sqrt{2}} (\theta^a + i \phi^a), \\
  \bar{m}^a &=& \frac{1}{\sqrt{2}} (\theta^a - i \phi^a),
\end{eqnarray}
%
where $\{t^a,r^a,\theta^a,\phi^a\}$ are the standard unit vectors of
Minkowski space in spherical coordinates. In particular,
$\{r^a,\theta^a,\phi^a\}$ is right-handed. In Cartesian coordinates,
%
\begin{eqnarray}
        t^a &=& (1,0,0,0), \\
        r^a &=& \frac{1}{r} (0,x,y,z), \\
        \theta^a &=& \frac{1}{\rho r} (0, x z, y z, - \rho^2), \\
        \phi^a &=& \frac{1}{\rho} (0, -y, x, 0),
\end{eqnarray}
%
where $\rho=\sqrt{x^2+y^2}$ and $r=\sqrt{x^2+y^2+z^2}$.

Note: If the metric is not flat, one should really compute something
like principal null directions of that metric, but typically one just
starts with the flat space tetrad and performs a Gram-Schmidt
orthonormalization. Commonly one begins this procedure with the $\phi^a$
vector. In summary, this procedure leads to the following operations
%
\begin{eqnarray}
  \phi^a   &\rightarrow& \frac{\phi^a}{\sqrt{\langle \phi,\phi\rangle}}, \\
  r^a      &\rightarrow& r^a - \langle r, \phi \rangle \phi^a, \\
  r^a      &\rightarrow& \frac{r^a}{\sqrt{\langle r,r\rangle}}, \\
  \theta^a &\rightarrow& \theta^a - \langle \theta, \phi \rangle \phi^a
                                  - \langle \theta, r \rangle r^a, \\
  \theta^a &\rightarrow& \frac{\theta^a}{\sqrt{\langle \theta, \theta
                         \rangle}},
\end{eqnarray}
%
where we have introduced the notation $\langle u, v \rangle = g_{ab}u^a v^b$.




\mysec{Symmetry properties of $\psi_4$}

$\psi_4$ is not a tensor but a ``pseudo-scalar''. The reason is that
although the Weyl tensor is a tensor, the definition of the tetrad
introduces non-tensor quantities. To be precise (is
this actually a complete characterization?), $\psi_4$ is a
pseudo-scalar with spin-weight $-2$. This spin-weight refers to
spatial rotations in the tetrad basis. (This should be spelled out!
What is the defining equation?)

The non-tensor property of $\psi_4$ has several consequences. If one
performs a mode decomposition of $\psi_4$, the appropriate basis
functions are not the usual spherical harmonics but the (far less well
known) spin-weighted spherical harmonics, see below.

Another issue of relevance in a numerical code is that $\psi_4$ does
not behave like a scalar under reflections and rotations of the
coordinates. Therefore certain symmetry boundary conditions have
to be modified, for example at the boundary of the numerical domain,
but also in the evaluation of integrals.



\mysec{Transformation of $\psi_4$ under reflections}

Notation: In a given chart on a manifold, a tensor $T$ transforms
under a diffeomorphism $f: p \rightarrow p' = f(p)$ as
\beq
        T'(p') = (J T)(p),
\eeq
where $p$ is a point, $f$ a function mapping points to points, $J$ 
the Jacobian of $f$, and $J T$ denotes the transformation of the
contravariant indices of $T$ by the matrix $J$ and covariant indices
by $J^{-1}$. A tensor is {\em symmetric} under this
transformation if for any point $q$
\beq
        T'(q) = T(q).
\eeq

Reflection: In Cartesian coordinates the reflection e.g.\ at the $z=0$ plane
is defined by
\bea
         p'&\equiv& (t',x',y',z') = (t,x,y,-z), \\
        {J^{a}}_b &\equiv& 
        \frac{\partial x'^{a}}{\partial x^b} = \mbox{diag}(1,1,1,-1).
\eea
In words, under reflections the components of a tensor pick up a minus
sign for each occurrence of a $z$-index, with immediate generalization
to reflections at the other coordinate planes.

For a vector $v^a$ that is symmetric under a given reflection, we have
\beq
 Jv^a(p) = v^a(p'), \quad \mbox{or equivalently}\quad Jv^a(p') = v^a(p). 
\eeq
While $t^a$ and $r^a$ transform like vectors, the components of
$\theta^a$ and $\phi^a$ in Cartesian coordinates transform as follows
under reflections:
\beq
\begin{tabular}{c|ccc}
           & $x\rightarrow -x$ & $y\rightarrow -y$ & $z\rightarrow -z$
\\ \hline
$J\theta^a(p')$ & $+\theta^a(p)$ &   $+\theta^a(p)$ &  $ - \theta^a(p)$ 
\\
$J\phi^a(p')$ & $-\phi^a(p)$ &   $-\phi^a(p)$ &  $ + \phi^a(p)$ 
\end{tabular}
\eeq
This implies that e.g.\ for the $z$-reflection
\bea
        J\bar{m}^a(p') &=& 
                 (J\theta^a(p') - i J\phi^a(p'))/\sqrt{2}
\\
             &=& (-\theta^a(p) - i \phi^a(p))/\sqrt{2}
\\
                       &=& - m^a(p),
\\
        \bar{m}^a(p') &=& - Jm^a(p),
\eea
that is, $\bar{m}^a$ is transformed into $m^a$ with a sign change and
vice versa, so $\bar{m}^a$ neither transforms as a vector nor as a
pseudo-vector with a sign flip.

We can now address the question how $\psi_4$ transforms under
reflections. By definition
\beq
        \psi_4(p) = C_{abcd} n^a \bar{m}^b n^c \bar{m}^d (p).
\eeq
Note that the pseudo-vector pieces are contained in 
\beq
        \bar{m}^a \bar{m}^b = (\theta^a\theta^b - \phi^a\phi^b)/2
                             - i (\theta^a\phi^b + \phi^a\theta^b)/2.
\eeq
This product has the property that for each of the $x$-, $y$-, or
$z$-reflections 
\bea
  \mbox{Re}(\bar{m}^a\bar{m}^b)(p') &=& + \mbox{Re}(J\bar{m}^aJ\bar{m}^b)(p),
\\
  \mbox{Im}(\bar{m}^a\bar{m}^b)(p') &=& - \mbox{Im}(J\bar{m}^aJ\bar{m}^b)(p).
\eea
In general, $\psi_4$ does not possess any symmetries even if the tetrad
has certain symmetries, since in general the Weyl tensor is asymmetric. 
Its transformation law is $C'(p') = J^{-1} C(p)$. Assuming
reflection symmetry, however, we have $C(p') = J^{-1} C(p)$. Together
with the reflection symmetries established above for the explicit
tetrad in flat space, we find
\bea
        \psi_4(p') &=& C_{abcd} n^a \bar{m}^b n^c \bar{m}^d (p')
\\
        &=& (J^{-1}C)_{abcd} (J n)^a (J n)^c 
            (\mbox{Re}(J\bar{m}^bJ\bar{m}^d) - 
            i\mbox{Im}(J\bar{m}^bJ\bar{m}^d))(p)
\\
        &=& C_{abcd} n^a n^c 
            (\mbox{Re}(\bar{m}^b\bar{m}^d) 
          - i\mbox{Im}(\bar{m}^b\bar{m}^d))(p).
\eea
In summary, assuming reflection symmetry of the Weyl tensor and using
the standard flat space null tetrad,
\bea
        \psi_4(p') &=& \overline{\psi_4(p)},
\\
        \mbox{Re}\psi_4(p') &=& +\mbox{Re}\psi_4(p),
\\
        \mbox{Im}\psi_4(p') &=& -\mbox{Im}\psi_4(p).
\eea
The real part of $\psi_4$ transforms as a scalar under all
three reflections, while the imaginary part of $\psi_4$ transforms as
a pseudo-scalar with a sign change.

Note: This argument has to be reexamined for any other tetrad that one
may want to use. Conceivably, the above symmetry property will not
hold if the tetrad is orthonormalized using a non-flat metric. Even if
the metric is reflection symmetric, orthonormalization introduces
linear combinations of the $\theta^a$ and $\phi^a$ vectors that may
not transform simply with an overall sign change in the imaginary
part. (CHECK: this actually does not seem the case, awaits explicit
formula for orthonormalization.)




\mysec{Spin-weighted Spherical Harmonics}

Collect all relevant formulas here. The Mathematica notebook Ylm.nb in
bam/src/math contains the derivation using Wigner d-functions and also
using ladder operators. For example,
%
\begin{eqnarray}
  {}_{-2}Y_{2-2} &=& \sqrt{\frac{5}{64\pi}}
                     (1 - \cos{\theta})^2 e^{-2i\phi},  \\
  {}_{-2}Y_{2-1} &=& -\sqrt{\frac{5}{16\pi}}
                     \sin\theta (1-\cos\theta) e^{-i\phi}, \\
  {}_{-2}Y_{20} &=& \sqrt{\frac{15}{32\pi}} \sin^2{\theta}, \\
  {}_{-2}Y_{21} &=& -\sqrt{\frac{5}{16\pi}}
                    \sin\theta (1+\cos \theta) e^{i\phi} \\
  {}_{-2}Y_{22} &=& \sqrt{\frac{5}{64\pi}}
                    (1 + \cos{\theta})^2 e^{2i\phi},
\end{eqnarray}
%
Note that $(1 + \cos{\theta})^2 = 4 \cos^4(\theta/2)$ and
$(1 - \cos{\theta})^2 = 4 \sin^4(\theta/2)$.



\mysec{Computing the mode decomposition of $\psi_4$}

Since $\psi_4$ is of spin-weight $-2$, its modes are defined with
respect to the spin-weighted spherical harmonics so that the result
is a scalar (check, is this really what happens?). We denote the modes
by
\beq
        \psi_4^{lm} = (\mbox{}_{-2}Y_{lm}, \psi_4).
\eeq
The scalar product is defined for two complex functions $f(\theta,\phi)$ and
$g(\theta,\phi)$ by
\beq
        (f,g) = \int_0^\pi d\theta \sin\theta \int_0^{2\pi} d\phi 
                \,\, \overline{f(\theta,\phi)} g(\theta,\phi).
\eeq

Reflection symmetries: Suppose only the upper half space is available
because the underlying metric data is reflection symmetric (bitant
symmetry). A $z$-reflection corresponds to
\beq
        \theta' = \pi-\theta, \quad \phi' = \phi,
\eeq
and 
\beq
        \mbox{}_{-2}Y_{22}(p') = \overline{\mbox{}_{-2}Y_{2-2}(p)}.
\eeq
As apparent from its definition, $\mbox{}_{-2}Y_{22}$ is asymmetric
as a function of $\theta$, being largest in the upper half space,
while $\mbox{}_{-2}Y_{2-2}$ is largest in the lower half space.

To compute $(\mbox{}_{-2}Y_{22}, \psi_4)$ we can split the integral
for an arbitrary function $h(\theta)$ according to
\beq
        \int_0^\pi d\theta \sin(\theta) h(\theta) =
        \int_0^{\pi/2} d\theta \sin(\theta) [h(\theta) +
        h(\pi-\theta)],
\eeq
since
\beq
        \int_{\pi/2}^{\pi} d\theta \sin\theta h(\theta) = 
        \int_{\pi/2}^0 d\theta' \sin(\pi-\theta') h(\pi-\theta') =
        \int_0^{\pi/2} d\theta \sin(\theta) h(\pi-\theta). 
\eeq
(Check signs). For example,
\beq
   (\mbox{}_{-2}Y_{22}, \psi_4) = 
   (\mbox{}_{-2}Y_{22}, \psi_4)_{\theta\in[0,\pi/2]} +
   \overline{(\mbox{}_{-2}Y_{2-2},\psi_4)}_{\theta\in[0,\pi/2]}.
\eeq


\mysec{ADM momenta}

The notion of the energy and momentum associated with a localized area or
a particular object represents a highly non-trivial problem
in general relativity.
The situation is less complicated if we talk about the total energy or
momentum associated with a spacetime described by the two fundamental
forms $\gamma{ij}$ and $K_{ij}$. Provided the fundamental forms satisfy
certain assymptotic requirements, the total energy and momentum
of the spacetime are finite and well-defined and can be obtained from
integrals over 2-spheres evaluated in the limit of infinite radii
(see, e.\,g.\,\cite{York1979}). These integrals are commonly referred
to as {\em ADM quantities} in the literature. For the ADM mass and linear
momentum we use the definition given in Wald \cite{Wald1984}:
%
\begin{eqnarray}
  E_{\rm ADM} &=& \frac{1}{16\pi} \lim_{r\rightarrow \infty} \int
     \left( \partial_m \gamma_{mn} - \partial_n \gamma_{mm} \right)
     N^n dA, \label{eqn:ADMenergy} \\
  P_n &=& \frac{1}{8\pi} \lim_{r\rightarrow \infty} \int \left(
     K_{mn} N^m - K^m{}_m N_n \right) dA. \label{eq: ADMP}
\end{eqnarray}
%
Here we assume summation over repeated indices (including pairs of
downstairs indices) and
$N^n$ represents the unit normal associated with the area element $dA$.
Similarly, we calculate the ADM angular momentum from
%
\begin{equation}
  J_a = \frac{1}{8\pi} \epsilon_{am}{}^n \int x^m A^l{}_n N_ldA,
\end{equation}
%
(see, e.\,g.\, , \cite{Yo2002}). Note that the last equation can be
rewritten as
%
\begin{equation}
  J_i = \frac{1}{8\pi} \epsilon_{am}{}^n \int x^m
     \left(K^l{}_n - \frac{1}{3} K \delta^l{}_n \right) N_ldA
      = \frac{1}{8\pi} \epsilon_{am}{}^n \int x^m K^l{}_n N_ldA,
      \label{eq: ADMJ}
\end{equation}
%
since the radially outgoing vector $N_l \sim x_l$.

Eqs.\,(\ref{eq: ADMP}), (\ref{eq: ADMJ}) can be expressed together in a
single equation in terms of the Killing vectors associated with the
momenta \cite{York1979}. Note that the term involving the trace of the 
extrinsic curvature $K$ is missing in Eq.\,(11) in \cite{Bowen1980},
presumably because they are explicitly dealing with maximally sliced
($K = 0$) solutions. 


\mysec{The ADM energy for conformally decomposed metrics}

[This is possibly superfluous detail, but this seemed the best place 
for it --- Mark Hannam.]

In practice we conformally decompose the spatial metric as 
%
\begin{equation}
\gamma_{ij} = \psi^4 \tilde{\gamma}_{ij}.
\end{equation}
%
In this case it is possible to write the ADM energy (\ref{eqn:ADMenergy}) in
a much simpler form. Explictly, using the definition in \cite{York1979} (which 
reduces to Eq.\,(\ref{eqn:ADMenergy}) in the flat-space limit at spatial
infinity), the energy is now 
%
\begin{eqnarray}
E_{\rm ADM} & = & \frac{1}{16\pi} \lim_{r\rightarrow \infty} \int
\sqrt{\gamma} \gamma^{ij} \gamma^{kl} 
(\nabla_l \gamma_{jk} - \nabla_j \gamma_{kl} ) N_i dA \\
& = & - \frac{1}{2\pi} \lim_{r\rightarrow \infty} \int
      \psi \sqrt{\tilde{\gamma}} \tilde{\nabla}^i \psi N_i dA + \nonumber \\
& & \frac{1}{16\pi} \lim_{r\rightarrow \infty} \int \psi^2
\sqrt{\tilde{\gamma}} \tilde{\gamma}^{ij} \tilde{\gamma}^{kl} 
(\tilde{\nabla}_l \tilde{\gamma}_{jk} - \tilde{\nabla}_j \tilde{\gamma}_{kl} ) N_i dA.
\label{eqn:ADMenergy_conf}
\end{eqnarray}
%
The covariant derivative $\nabla_i$ is with respect to the physical spatial metric 
$\gamma_{ij}$ and $\tilde{\nabla}_i$ is with respect to the conformal metric 
$\tilde{\gamma}_{ij}$. In the limit $r \rightarrow \infty$, the
conformal factor is unity. In addition, if 
$\tilde{\gamma}_{ij} = O(r^{-2})$ then the second
integral will not contribute. 

York \cite{York1979} and O'Murchadha and York \cite{omurch74} discuss
this situation. In section 9.4 of \cite{York1979} York describes a 
``quasi-isotropic'' gauge condition that reduces to precisely the requirement 
that we want: $\tilde{\gamma}_{ij, k} = O(r^{-3})$. Conformally flat 
Bowen-York data trivially satisfy this requirement, but it is also 
satisfied for other possible data sets, for example the conformal metric 
of Kerr in quasi-isotropic coordinates. 

More importantly, York notes that if we evolve with a minimal distortion shift, 
then this condition is preserved in an evolution. This can be seen also
for the gamma-driver shift conditions. The right-hand side of the metric evolution 
equation involves the extrinsic curvature and derivatives of the shift vector.
At large radii both fall off as $1/r^2$, and so if 
$\delta_{ij} - \tilde{\gamma}_{ij} = O(r^{-2})$ on the initial slice, it will
remain $O(r^{-2})$ in an evolution. 

%O'Murchadha and York go into more detail, although they do not mention the
%``quasi-isotropic'' gauge. In the sections around 
%Eq.\,(18) in \cite{omurch74} they show that we need only the first 
%term in (\ref{eqn:ADMenergy_conf}) if our space is asymptotically conformally
%flat, presumably approacing conformal flatness at the rate described above ---
%the conformal metric used in BSSN evolutions of Kerr-Schild data, for example,
%do not satisfy this requirement. Our initial data is conformally flat everywhere, 
%and since  deviations from conformal flatness propogate at finite speed (we hope!),
%the space will always be asymptotically conformally flat. 

Keeping only the first term in Eq.\,(\ref{eqn:ADMenergy_conf}), the ADM energy is 
 simply 
%
\begin{equation}
E_{\rm ADM} = - \frac{1}{2\pi} \int_{\infty}
 \partial_i \psi N_i dA, \label{eqn:ADMenergy_simple}
\end{equation}
%
where we have used the fact that at spatial infinity 
$\tilde{\gamma} \rightarrow 1$, and the covariant derivative of 
the scalar $\psi$ can be written as a partial derivative. Note
also that the conformal factor $\psi$ can be replaced in 
Eq.\,(\ref{eqn:ADMenergy_simple}) by the BSSN variable $\phi$ 
--- in the $r \rightarrow \infty$ limit the two expressions will be 
equivalent. This is what is done in the BAM implementation in the 
ADM\_mass project, where ``ADM\_mass'' denotes the mass from 
Eq.\,(\ref{eqn:ADMenergy}), and ``ADM\_mass (b)'' denotes the
mass from Eq.\,(\ref{eqn:ADMenergy_simple}).

In practice we evaluate this integral at some finite radius, which
will introduce an error. It remains to be seen how this integral
converges with increasing radius as compared with (\ref{eqn:ADMenergy}). 

Note also that (\ref{eqn:ADMenergy_simple}) may be easily converted
via Gauss's theorem into a volume integral. If we do this, we should
think twice before replacing the Laplacian of $\psi$ with the
rest of the Hamiltonian constraint --- the constraints are not strictly
satisfied in a numerical evolution! Of course, we hope that they converge
to zero, and having thought twice we might decide that it is Ok to use
them after all. 

\bibliographystyle{plain}
\bibliography{references.bib}


\end{document}
 
