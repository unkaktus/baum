\documentclass[a4paper,10pt]{article}

% --------------------------------------
% Packages
\usepackage{graphicx}
\usepackage{amsmath}
\usepackage{amssymb}
\usepackage{color}
\usepackage{comment}

% --------------------------------------
% New commands/macros
\newcommand{\be}{\begin{equation}}
\newcommand{\ee}{\end{equation}}

% --------------------------------------
% Title page setup
\title{Hydro analysis}
\author{S.Bernuzzi \texttt{sebastiano.bernuzzi@uni-jena.de}}
\date{\today}

% ++++++++++++++++++++++++++++++++++++++

\begin{document}

% --------------------------------------
% Title page, abstract, and table of contents
%\maketitle
%\begin{abstract}
%\end{abstract}
%\tableofcontents

% --------------------------------------
\section{Equations}
\label{sec:eqs} 

\subsection{Bounded/unbounded rest-mass}

Given the rest-mass density $D$, the bounded (unbounded) part is
determined as  
\be
- u_t > 1 \ \ \ (-u_t < 1 )
\ee 
where $u^a$ is the 4-velocity of the fluid, 
\be
u^a = W\,( n^a + v^a) 
= W \left[ 1/\alpha , v^i - \beta^i / \alpha \right] \ .
\ee
The volume integrals
\be
M = \int d^3x D
\ee 
give the rest-mass. Once the rest-mass densities
(bounded/unbounded/outside the horizon) are computed, standard scalar
output routines can be used to compute the related masses. 
(\verb output_0d/*_integral.l? ).

\subsection{Vorticity}

The vorticity tensor is defined as
\be
w_{ab} = \partial_a (h u_b) - \partial_b (h u_a) \ ,
\ee
where 
\be
h = 1 + \epsilon + \frac{p}{\rho}
\ee
is the specific enthalpy.
In practice the components $w_{ab}$ define a vector $w_i$ and we
compute 
\be
\label{eq:vor}
w_z = \partial_x (h u_y) - \partial_y (h u_x) \ ,
\ee
and permutations.

% --------------------------------------
\section{Parameters and variables}
\label{sec:parvar} 

Parameters
\begin{itemize}
\item \verb#hydroanalysis_D# the variable name to be used for $D$ (\verb#grhd_D#)
\item \verb#hydroanalysis_p# the variable name to be used for $p$ (\verb#grhd_p#)
\item \verb#hydroanalysis_r# the variable name to be used for $\rho$ (\verb#grhd_rho#)
\item \verb#hydroanalysis_e# the variable name to be used for $\epsilon$ (\verb#grhd_epsl#)
\item \verb#hydroanalysis_v# the variable name to be used for $v^i$ (\verb#grhd_vx#)
\item \verb#hydroanalysis_lmin# analysis of levels $l>l_{\rm min}$
\item \verb#hydroanalysis_lmax# analysis of levels $l>l_{\rm max}$
\end{itemize}
Variables
\begin{itemize}
\item \verb#hydroa_Dh# $D$ outside the horizon
\item \verb#hydroa_Db# $D$ bounded 
\item \verb#hydroa_Du# $D$ unbounded 
\item \verb#hydroa_vor[xyz]# vorticity
\end{itemize}

% --------------------------------------
\section{Tests}
\label{sec:tests} 

\verb#par/star.par#\\
For a static star one has
\begin{itemize}
\item $D^{\rm b} = D$ $D^{\rm u} = 0$ $D^{\rm h} = 0$ 
\item $w_i=0$
\end{itemize}


% --------------------------------------
\section{Log changes/TODO list}
\label{sec:log} 

Please include here your changes and the svn revision.

\begin{itemize}


\item 19.11.2013 r9742
    \begin{itemize}
  \item Include computation of total Energy and total internal energy, according to arXiv:1212.0905
  \item Safety check, ejecta is now only outward moving matter.
  \item Include Mode projection of rho on levels, the origin for the spherical harmonics is always the origin
  \item TODO. Decide whether we want to have the origin at the puncture positions to meassure the modes of the individual star during the inspiral.
  \end{itemize}

\item 7.02.2013 r8693
  \begin{itemize}
  \item Use atmosphere to exclude some points from computation 
  \item TODO. Skip the analysis in shells and in levels $l<l_{\rm min}$ and $l>l_{\rm max}$, see comments in \verb#hydroanalysis.c#   
  \item TODO. Standard benchmark with a binary (include a parfile).
  \end{itemize}

\item 5.02.2013 r8693
  \begin{itemize}
  \item Fixed calculation of Lorentz factor $W$ and added safe divisions in \verb#compute_hydrovars.m#   
  \item Added OpenMP support in \verb compute_hydrovars.m  , to be tested.
  \item Set to 0 all the rest-mass density inside the horizon.
  \item TODO. Skip the analysis in shells and in levels $l<l_{\rm min}$ and $l>l_{\rm max}$, see comments in \verb#hydroanalysis.c#   
  \item TODO. Standard benchmark with a binary (include a parfile).
  \end{itemize}

\item 30.10.2012 r8046
  \begin{itemize}
  \item First version of the project.
  \item Derivatives in Eq.~\eqref{eq:vor} are taken with operator
    $D_0$. If it is oscillatory, a non-oscillatory derivative may be
    preferable. 
  \item TODO. Include Entropy calculation. 
    For ideal gas $s=p/\rho^\Gamma$, Is there a
    simple and general diganostic for entropy?
  \end{itemize}

\end{itemize}




% --------------------------------------
%\begin{thebibliography}{100}

%\end{thebibliography}
% --------------------------------------

% ++++++++++++++++++++++++++++++++++++++

\end{document}
